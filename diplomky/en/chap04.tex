\chapter{Evaluation}


\label{chap:eval}

Until now we showed various possibilities of constructing a model.
In this chapter we will describe how to compare the performance of models.
We cannot mathematically prove that a model is a correct estimation of what we are trying to classify.
This is because we do not know the actual model. \jh{Tohle je trochu tezko pochopitelne - vy myslite ten idenalni model. (1) asi bych pridal slovo "real" (the actual real) a 
(2) misto model bych pouzil slovo "distribution"}
Therefore, we will use statistical methods on a testing set.

We cannot validate a model based on how well it fits our training data, because that would yield over-optimistic results.
Also, it wouldn't give us any idea how well the classifier performs on data it has not seen during the training phase.
Hence, we will create a special testing set, which has not been seen by the algorithm during the training phase.
We will also introduce different metrics for expressing properties of the model. \jh{tady je matouci to dvoji "will", kazde se vyjadruje k necemu jinemu - jedno k tomu jak funguje evaluace a druhe jak je ji popisujete v teto kapitole, proto bych misto prvniho will bych dal "need" a misto "create" bych dal "choose" or "select": Hence, we need to choose a special ...}

\section{Overfitting}

In \jh{the (anglictinu vam tam ale urcite zkontroluje nekdo povolanejsi)} previous chapter \todoA{ref} we discussed that a model has to generalise well. This means it has to be able to predict even
instances it has not seen during the training phase. When a model fits well training data, but does not generalise well, it is called
{\bf overfitting} (sometimes referred to as {\bf bias}).
This happens when an algorithm learns to recognise idiosyncrasies of the training set, rather than to understand the underlying logic.
These idiosyncrasies can appear by incomplete data or noise introduced during obtaining the data.

One way of thinking about it describes a phenomena \jh{phenomenon} known as {\it occam's \jh{Occam's} razor}. \todoA{citation}
It basically advice\jh{s} us to prefer \jh{a} less complicated solution over the \jh{a} more complicated \jh{one} as it is less likely to suffer from overfitting.
An example of such a process can be a {\bf Decision Tree}.
If we have two trees with comparable performance, the one with less nodes is less likely to suffer from overfitting. \citet{TanBachKum08} describe this in chapter 4.

{source: kumar 188}

\todoA{add mention of underfittin???}

\section{Partitioning Data}

Usually, we do not get data divided into training and testing sets.
Even though it gives us more flexibility, it also introduces a new challenge how to obtain the two sets.
\j{Tady neco schazi. Reknete, ze jsou dva hlavni pristupy, jak to delat - ...., Below we describe them in more detail.}

\subsection{Holdout Method}

The simplest approach is to randomly divide data into two sets, such that their size will be of a given ratio.
This approach is called {\bf Holdout}.
The exact ratio is at the discretion of further analysis and usually ranges between 50 and 70~\% for the training set. \todoA{citation}. \jh{Je to dulezity? Spis bych rekl, ze je treba vybalancovat dva protichudne pozadavky - mit co nejvice dat na trenovani, aby byl model co nejlepsi, a zaroven mit dostatek testovacich dat, aby byla evaluace spolehliva. }

As \citet{TanBachKum08} note there are three issues with this approach.
First, having fewer training examples can lead to lower performance.
Also, the model can be depended on the actual subset chosen for training.
Second, If \jh{if} we use too much data for training, the performance estimation \jh{pouzijte slovo evaluation, ktere jste pouzival doted} will be less reliable.
Statistically speaking, it will have wider confidence interval.
Finally, the training and testing sets are not independent of each other anymore.
An example of this dependency is when a class is overrepresented in one subset, it must be underrepresented in the other one.
\jh{POslednimu budu nerozumim presne}

\subsection{Cross-Validation}

To address some of the issues listed above, {\bf Cross-Validation} has been introduced.
The underlying idea is that each instance is used the same number of times for training and exactly once for testing.
We split the dataset into~$k$~equally-sized subsets.
For each subsets, we train our algorithm on all subsets but the one chosen and the one will be used as a testing set.
This approach is called {\bf k-fold cross-validation} for~$k$~subsets. The variable~$k$~is chosen arbitrarily.

Special case for~$k=N$, \jh{myslim, ze bez carky - je to tesny privlastek} where~$N$~is the number of instances is called {\bf leave-one-out} and has the advantage of 
utilizing as much data for training as we can.
However, it is computationally expensive and we therefore generally choose lower~$k$~for practical uses. \jh{uses $>$ reasons?}

For evaluation we compute evaluation metrics for each pair \jh{of} training and testing set\jh{s} separately and report their average along with minimal and maximal value and standard deviation. \jh{tady to vypada, ze tak se to dela vzdycky, a pritom rikate, ze tak to delate v teto praci. Reknete to: In this thesis, we compute evaluation ...}

\todoA{should I elaborate more on this statistics?}


\section{Evaluation Metrics}

When we have our training and testing sets, we can train our classifier.
Subsequently, we predict labels of instances in the testing set with our newly
built classifier.
To say how well the classifier performs, we need to express how well the predicted
labels fit the actual labels.
For doing this we will introduce several {\bf evaluation metrics}.

Let us have a binary classification problem and a classificator we want to evaluate.
For evaluation we always have to decide which label we will evaluate it on.
In our case we will choose ``useful''.
All instances that our classifier predicted as this label are then {\bf positive} and the rest is {\bf negative}.
Both categories are further divided into true and false.
True being instances that our classifier predicted correctly and false incorrectly.
Together we have four categories;
true positive, false postive, true negative and false negative.
For clarity we can arange these four categories into the so-called {\bf confusion matrix}.
Each cell represents the number of instances from one of the categories.
First and second columns correspond to positive and negative instances, respectivelly.
Rows to true and false.
%then matrix~$A$~of the dimensions 2x2 is a {\bf confusion matrix} if and only if~$A_{1,1}$~and~$A_{1,2}$~are the numbers of correctly and incorrectly classified positive instances, respectively. $A_{2,1}$ and~$A_{2,2}$~are then analogically correctly and incorrectly classified negative instances.
In the example below, 2 not-useful instances have been classified as useful - these are false positives.

\todoA{example}
\begin{code}
        classified    useful       not
actual
useful                TP 10        FN  5
not                   FP  2        TN  11
\end{code}


The most straightforward metrics are accuracy and error rate.
{\bf Accuracy} is the ratio of correctly classified instances and {\bf error rate} incorrectly.
In the confusion matrix it is then:


\todoA{formula}

accura = TP + TN / sum

error\_rate = FP + FN / sum

In our toy example accuracy is~$21/28$ and error rate~$7/28$.

\todoB{problems with accuracy}.

More commonly used metrics for their better properties are precision, recall and f-measure.
{\bf Precision} is the ratio of correctly classified instances out of those classified as positive.
It expresses our confidence on how likely an instance is positive when we classify it so.

precision = Tp / tp + fp

{\bf Recall} is the ratio how many instances we classified as positive out of those that are truly positive.
It expresses our confidence on how many instances we revealed out of the positive cluster.

recall = tp / tp+ fn

In our toy example precision and recall are 10/15 and 10/12 respectively.

To combine recall and precision {\bf f-measure} is commonly used.
It is harmonic average of the two and expresses combination of both properties.

\todoA{formula}


{source: KUMAR - 150, 550}




\subsection{\todoC{guyon - kniha -a pendix A hezky prehled null hypothesis}}

\subsection{\todoB{error analysis - see w8}}

How to debug and improve from what we saw.

\subsection{\todoB{learning curves - overfitting/underfitting}}
