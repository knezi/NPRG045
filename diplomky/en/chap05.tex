\chapter{The Software Project Architecture}
\label{chap:arch}

The software package was designed and developed with extensibility in mind.
It allows to add new features, specify different subsets of data and convert them into various format without breaking older classifications and experiments.
Classifiers and preprocessing methods are implemented children of abstract classes
to allow to easily add new algorithms.
The entire experiment is configured in a special YAML\footnote{\url{https://yaml.org}} file.
It allows precise definition of conducted experiments and plotted graphs.
New folder for statistics is created every run to prevent accident overwritting of already obtained results.
The entire project is written mostly in Python of version at least 3.6\footnote{\url{https://python.org}} and bash\footnote{\url{https://gnu.org/software/bash}} with high usage of libraries for both higher performance and easier mantainability.
All classes and files have documentation strings which should be sufficient
for detailed understanding.
In this chapter, we describe the high-level overview.

In \autoref{tab:files}, there are defined a few names for different files for easier understanding.

\begin{table}[h]

\centering
\begin{tabular}{ll}
\toprule
\textbf{name of file}& \textbf{purpose} \\
\midrule
instance file		 & stores instances in the final format \\
intermediate file	 & used for generating Geneea file \\
Geneea file			 & contains extra linguistics data for instances \\
experiments file	 & configures the experiments conducted \\
\bottomrule
\end{tabular}

\caption{File definition}\label{tab:files}
\end{table}

\todoA{add libraries used - as another appendix??}


\section{Experiments File}

This file contains entire configuration of experiments conducted.
It lists classification tasks.
Each task is a definition of a classifier to be trained and evaluated.
Next, it contains list of graphs with results to be plotted.

Experiments file has the following format.
The root element is a dictionary with three parts;
\texttt{config}, \texttt{tasks} and \texttt{graphs}.
An example file can be found bellow in \autoref{lst:yamlfile}.

\begin{lstlisting}[label={lst:yamlfile}]
config:
  chunks: 10
tasks:
  - name: 'zero-R'
    classificator: 'baseline'
    features:
      - REVIEWLEN
      - UNIGRAMS
    preprocessing:
      - 'mutualinformation'
      - 'featurematrixconversion'
    extra_data: []
    config:
      algorithm: 'zero-R'
      features_to_select: 2
graphs:
  - name: 'baseline'
    data:
      zero-R:
        - 'f-measure'
\end{lstlisting}

The config section is another dictionary.
It has only one element -- \texttt{chunks}.
It specifies into how many partitions the dataset set is split for k-fold cross-validation.

The section experiments is a list of classification tasks to be conducted.
Each element is a dictionary specifying the exact parameters.
Descriptions of individual fields is in \autoref{tab:exp_dict}
An example of \texttt{extra\_data} is text of a review if the classifier needs to work
with the orignal text.

\begin{table}[h]

\centering
\begin{tabular}{lll}
\toprule
\textbf{element name} & \textbf{type} & \textbf{meaning}\\
\midrule
name 			& string	& identification referred to in graphs\\
classificator 	& string	& used classifier\\
features 		& list		& extracted features \todoA{reference} \\
preprocessing 	& list		& applied preprocessors in order\\
extra\_data 	& list 		& extra attributes passed to the first preprocessor \\
config			& dict		& extra configuration for all parts of pipeline \\
\bottomrule
\end{tabular}

\caption{Experiment Configuration}\label{tab:exp_dict}
\end{table}


The last section \texttt{graphs} is used for specifying what graphs will be plotted.
It is a list of individual figures.
Each figure is a dictionary of two elements; \texttt{name}, \texttt{data}.
Name specifies the filenames of the resulting graph ---
for each figure png and csv files will be created.
Element \texttt{data} specifies what data will be used.
It is a dictionary of experiment names, values being a list of all metrics to be dumped.

\section{Parts Overview}

Entire project is executed by make\footnote{url{https://www.gnu.org/software/make}}.
It is split into two main targets;
Data preparation and running experiments.
The former is defined as targets `data/geneea.json` and `data/data.json`.
It is a direct dependency of the target `run`, which serves for generating results
of experiments.

\begin{code}
DIAGRAM BETTER
prepare data
	denormalize
	join
	compute geneea
	spellcheck

run experiments
	load data, gen samples, convert and prep
	process YAML
		- experiments
			- task by task
				- preprocess by prep
				- classify
				- evaluate
		- visualise
			- graph by graph (stats, graphs)
\end{code}


\section{Data Preparation}

Preparation itself can be run by:

\begin{code}
make data/data.json
\end{code}

All scripts for preparing data are in the folder \texttt{denormalization}.
It prepares instance and intermediate files.
The process is executed by:

\begin{code}
./denormalization/denormalize.sh ../data/dataset data/data.json data/ids
\end{code}

The first argument is a directory with extracted archive of the Yelp Dataset.
The second argument is path to instance file and the last one to intermediate file.
The script itself joins reviews and business
and filters out reviews outside the time frame we need.
The result is written into the instance file.
Next, it prepares list of all IDs of used reviews into the intermediate file one ID per line.
This file is processed on a separate server and the resulting geneea file must be stored
in \texttt{data/geneea.json}.
It contains additional linguistics data.
The exact description of files can be found in \autoref{app:a}.


\section{Running Experiments}

Experiments can be run by:

\begin{code}
make run
\end{code}

This will execute \texttt{process\_data.py}.
It loads all data from the instance and Geneea file.
Subsequently, it conducts all experiments as specified in the given experiments file.

\subsection{Load Data}

Class \texttt{Data} defined in file \texttt{load\_data.py} is used for loading data.
It stores raw data in Pandas DataFrame to achieve easier manipulation without compromising performance.
Whenever new sample is generated, the raw data is filtered out and split
into chunks.
The chunks are stored in an instance of \texttt{Sample}.
\texttt{Sample} still keeps only raw data.
Anytime a dataset is accessed all features are regenerated to ensure it is up-to-date.


\subsection{Conduct Experiments}

Subsequently, tasks are performed.
All tasks are guaranteed to be run on the exact same data to ensure comparability of results.
Both training and testing is implemented by pipeline of preprocessors and a classifier.
The preprocessors are in the same order as in the \textit{experiment} file.
A preprocessor must be defined in directory \texttt{preprocessors} and a child of \texttt{preprocessors/preprocessingbase.py}.

Every preprocessor has a method process, which must take data and return processed data.
The format of accepted and returned data arbitrary, so long the returned is compatible
with the one accepted by the next in the pipeline.
The last preprocessor must of course return format compatible with the classifier.
Hence, for most classifiers, it is needed to run conversion to matrix (\texttt{featurematrixconversion.py}).
The first preprocessor in the pipeline must take feature\_dict with additional data as returned by \texttt{load\_data.Data.get\_feature\_dict}.

The pipeline is the same for both training and testing data with the guarantee that
training data is always passed first and classifier is called train and classify for
training and testing data, respectively.

After all tasks are run, results are plotted and dumped.
All necessary logic is defined in \texttt{statistics.py}.
For storing performance data, class \texttt{DataGraph} is used.
It gathers data and publishes API for data manipulation.
An instance of this class is subsequently passed to \texttt{Statistics},
which handles the actual data plotting and writting to files.



