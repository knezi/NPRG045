\chapter{Package Technicalities}\label{app:techn}

The entire project runs in Python of version at least 3.6\footnote{\url{https://python.org}} and bash.\footnote{\url{https://gnu.org/software/bash}}
The following libraries are used:

\begin{table}[h]
\centering
\begin{tabular}{ll}
\toprule
\textbf{library} & \textbf{purpose} \\
\midrule

nltk & tokenizer, Na\"{i}ve Bayes\\
pandas & DatFrame for storing raw data in memory\\
scipy & sparse matrix for storing features\\
yaml & parser for configuration file \\
geneea-nlp-client & parser for loading Geneea analysis\\
gensim & compute cosine similarity \\
fasttext & classification model \\
aspell & spell checking \\


\bottomrule
\end{tabular}

\caption{Used libraries}\label{tab:libs}
\end{table}

The project is executed with make.\footnote{\url{https://www.gnu.org/software/make}}
By default, it runs all experiments files defined in directory \texttt{experiments}.
The entire project can be run by:

\begin{code}
make run
\end{code}

Or for demonstration running with a small subset of data:

\begin{code}
make run_sample
\end{code}

Because the NLP analysis is carried out remotely, \texttt{make} will prompt user for copying that file when appropriate.

Results are stored in \texttt{graphs/\textit{timestamp}}.
A new folder for statistics is created every run to prevent accidental overwriting of already obtained results.


\section{Experiments File}

The entire configuration of experiments is specified in a separate file in YAML format.\footnote{\url{https://yaml.org}}
It lists classification tasks and specifies graphs to be plotted.
Each task is a definition of a classifier.
Evaluation element is a definition what properties should be plotted in a graph.

Experiments file has the following format.
The root element is a dictionary with three parts;
\texttt{config}, \texttt{tasks} and \texttt{graphs}.
An example file can be found bellow.

\begin{code}
config:
  chunks: 10
  mi: true
  l_curves: false
  max_ngrams: 100
  max_tfidf: 100
tasks:
  - name: 'zero-R'
    classificator: 'baseline'
    features:
      - REVIEWLEN
      - UNIGRAMS
    preprocessing:
      - 'mutualinformation'
      - 'featurematrixconversion'
    extra_data: []
    config:
      algorithm: 'zero-R'
      features_to_select: 2
graphs:
  - name: 'baseline'
    data:
      zero-R:
        - 'f-measure'
\end{code}

The config section is a dictionary.
It must contain the elements specified in \Cref{tab:config_yaml}.

\begin{table}[h]
\centering
\begin{tabular}{lll}
\toprule
\textbf{key} & \textbf{datatype} & \textbf{value} \\
\midrule

chunks & int & parameter $k$  k-fold crossvalidation\\
mi & bool & if mutual information should be calculated\\
l\_curves & bool & ordinary/learning curves evaluation\\
max\_ngrams & int & number of unigrams used \\
max\_tfidf & int & number of words for TF-IDF used\\

\bottomrule
\end{tabular}

\caption{Config section}\label{tab:config_yaml}
\end{table}


The section \texttt{experiments} is a list of classification tasks.
Each element is a dictionary specifying the exact parameters.
Descriptions of individual fields is in \Cref{tab:exp_dict}.
The list \texttt{features} consists of names as defined in FeatureSetEnum.
The list of features is in \Cref{sec:data_stats}.
\texttt{extra\_data} is any property of instances available in the raw data.
An example is the original text or business attributes.

\begin{table}[h!]

\centering
\begin{tabular}{lll}
\toprule
\textbf{element name} & \textbf{type} & \textbf{meaning}\\
\midrule
name 			& string	& identification referred to in graphs\\
classificator 	& string	& used classifier\\
features 		& list		& used features \\
preprocessing 	& list		& applied preprocessors in the order\\
extra\_data 	& list 		& extra attributes passed to the first preprocessor \\
config			& dict		& extra configuration for all parts of pipeline \\
\bottomrule
\end{tabular}

\caption{Experiment Configuration}\label{tab:exp_dict}
\end{table}


The last section \texttt{graphs} is used for specifying what graphs will be plotted.
It is a list of individual figures.
Each figure is a dictionary of two elements; \texttt{name} and \texttt{data}.
Name specifies the filenames of the resulting graph ---
for each figure png and csv files will be created.
Element \texttt{data} specifies what evaluation metrics for what classifiers will be used.
It is a dictionary of experiment names, values being a list of all metrics to be output.


\section{Defining New Preprocessors and Classifiers}

A preprocessor must be defined in directory \texttt{preprocessors} and be a child of \texttt{preprocessors/preprocessingbase.py}.
Also, its name must be added into \texttt{preprocessors/\_\_init\_\_.py}.
Analogously, a new classifier is to be created in directory \texttt{classifiers}.

The input and output formats of preprocessors are arbitrary, so long the following conditions are met:

\begin{itemize}
	\item  adjacent units must be compatible.
	\item the first preprocessor in the pipeline must take \texttt{feature\_dict} as returned by \texttt{load\_data.Data.get\_feature\_dict}.
	\item the last preprocessor must be compatible with the input of the used classifier.
\end{itemize}

The input of the classifier is up to the user.
However, the output of the classifier must be the predicted label.
